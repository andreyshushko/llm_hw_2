\section{Метод Ньютона}

Предполагаем, что минимизируемая функция дважды дифференцируемая. Поэтому берем квадратичную аппроксимацию
\begin{equation*}
    x^{k  +1} = \argmin \limits_{x \in \mathcal{X}} \set*{f(x^k) + \mean*{\nabla f(x^k), x - x^k} + \frac{1}{2} \mean*{x - x^k , \nabla^2 f(x^k) (x - x^k)}}.
\end{equation*}
Получим явный вид для метода Ньютона на $\R^n$. Для этого просто берем производную и приравниваем ее 0: 
\begin{equation*}
    \nabla f(x^k) + \nabla^2 f(x^k) (x - x^k) = 0 \implies x^{k + 1} = x^k - \frac{\nabla f(x^k)}{\nabla^2 f(x^k)}.
\end{equation*}
Для квадратичной функции мы получим, что разложение в ряд Тейлора будет точным, таким образом мы получим сходимость в один шаг, ибо метод Ньютона даст точный ответ.













