\newpage
\section{Консультация Е.С.Половинкина}

\subsection{Дифференцируемость функций комплексного переменного. Условия Коши -- Римана. Интегральная теорема Коши.}

Параграф \S3, начало параграфов \S4 и \S5. 

Дать понятие производной функции. Доказать теорему: функция дифференцируема в точке тогда и только тогда, когда она дифференцируема как функция двух переменных и выполнено условие Коши -- Римана.

Понятие и свойства интеграла не требуются (но знать формулировки конечно нужно). 

Рассказать про лемму Гурса: есть некоторая область $G$ комплексной плоскости и замыкание треугольной области $\Delta$. Если $\Delta \subset G$, а функция $f$ -- регулярна внутри $G$, то интеграл $f$ по границе $\Delta$ равен 0. В доказательстве этой теоремы важно \textit{доказать}, почему интеграл равен сумме четырех интегралов, получаемых при делении треугольника.

Далее доказать интегральную теорему Коши. Это основная теорема всего курса! Все последующие теоремы, включая теорему о вычетах и теорему Коши -- Адамара, -- все они являются следствиями из этой теоремы.

Про границу и звездные области можно не доказывать.

\subsection{Интегральная формула Коши. Разложение функции, регулярной в окрестности точки, в ряд Тейлора.}

Параграф \S8, начало параграфа \S9.      

Если задана область специального вида $G$ и функция $f$, то значение функции в точке равняется 
\begin{equation*}
    f(z_0) = \frac{1}{2\pi i}\int_{\Gamma} \frac{f(z)}{z - z_0}dz
\end{equation*}
Это нужно доказать (напомним, что это делается через рассматривание точки на границе и ее окрестности, а дальше примененением теоремы Коши).  
Сказать про бесконечную дифференцируемость функции из интегральной формулу Коши (доказывать не нужно). Сказать, что отсюда следует бесконечная дифференцируемость регулярной функции.

Дальше про ряд Тейлора: с одной стороны по тоереме Абеля ряд в своей области сходимости (которая обязательно является кругом) является регулярной функции.

Возможный доп. вопрос: ряд сходится в некотором квадрате. Есть ли другие точки, в которых ряд сходится? Ответ: да, поскольку область сходимости всегда круг, можно описать круг около квадрата, и ряд будет сходится там.

Привести доказательство разложения регулярной в круге функции в сходящийся в этом круге ряд Тейлора. Доказательство, как водится, через интегральную теорему Коши.

\subsection{Разложение функции, регулярной в кольце, в ряд Лорана. Изолированные особые точки однозначного характера.}

Параграфы \S11 и \S12.

Вопросы объемные, поэтому доказывать достаточно только разложение в ряд Лорана.

Дать определение ряда Лорана. Сказать, что в любом замкнутом множестве кольца ряд Лорана сходится равномерно (локально равномерная сходимость). 

Теорема: если есть функция, регулярная в кольце, то она представима в виде сходящегося ряда Лорана. Доказательство практически аналогично доказательству теоремы для ряда Тейлора. Можно сформулировать единственность разложения, совпадение с рядом Тейлора; доказывать это в рамках билета не нужно.

Теорема о классификации особых точек (по главной части ряда Лорана -- без доказательства). Полюса, существенно особые и неустранимые особые точки.

\subsection{Вычеты. Вычисление интегралов по замкнутому контуру при помощи вычетов.}

Вычисление интегралов по замкнутому контуру -- очередное следствие интегральной теоремы Коши: если функция регулярна в области $G$ кроме, может быть, конечного числа точек, то интеграл по контору этой области $\Gamma$ равен
\begin{equation*}
    \oint_{\Gamma} f(z) dz = 2\pi i \sum_{z \in G} \mathrm{res}\, f(z).
\end{equation*}
Это надо доказать.

Сформулировать лемму Жордана о вычислении интегралов, доказать её.

С точки зрения Евгения Сергеевича, вопросы про комфорные отображения, регулярные ветви и другие поздние темы курса ТФКП задавать не должны, имеющихся вопросов достаточно. Конечно, знать это лишним не будет.

\subsection{Дополнительные вопросы}

Возможный дополнительный вопрос: если устремить точку $z$ внутри области и вне области к границе области, чему будет равен предел? Ответ: мы такое не изучали. На самом деле вопрос очень сложный, происходит определенный скачок, и изучениям этого скачка посвящено множество возможных статей.

Теорема Вейерштрасса: если сумма ряда сходится локально равномерно в области, то его можно почленно дифференцировать в этой области и полученный ряд будет сходиться локально равномерно.

Как найти производную функции действительного аргумента сразу со вторым порядком точности? Ответ: сделать из неё функцию комплексного переменного! Смотрим:
\begin{equation*}
    f(x + iy) = f(x) + iyf'(x) - y^2f''(x) + O(y^2).
\end{equation*}
Отсюда
\begin{equation*}
    f'(x) = \lim_{y \to 0} \frac{\mathrm{Im}\, f(x + iy)}{y}
\end{equation*}
-- второй порядок относительно $y$.




