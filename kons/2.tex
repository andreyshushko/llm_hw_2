\newpage
\section{Консультация О.К.Подлипского}

\subsection{Прямые и плоскости в пространстве. Формулы расстояния от точки до прямой и плоскости, между прямыми в пространстве. Углы между прямыми и плоскостями.}

Сначала нужно сформулировать ВСЁ про прямые и плоскости (причем важно сделать это для ОНБ или произвольной системы), записать их уравнения и продемонстрировать переходы между уравнениями.

В качестве доказательства берем формулу расстояния от точки до прямой и плоскости:
\begin{equation*}
    (\vecl{r} - \vecl{r}_0, \vecl{n}) = 0
\end{equation*}

Далее можно посчитать расстояние между прямой и плоскостью. Этого должно быть достаточно.

Дополнительно можно вспомнить лемму: вектор $(\alpha, \beta, \gamma)$ параллелен плоскости $Ax + By + Cz + D = 0$ тогда и только тогда, когда скалярное произведение направляющей и нормали равно 0 или, что то же самое, $A\alpha + B\beta + C\gamma = 0$; и доказать её, если есть время.

\subsection{Кривые второго порядка, их геометрические свойства.}

Нужно ввести параболы, гиперболы и эллипсы, рассказать про асимптоты и директриссы, вывести расстояние от точки до фокусов. Дальше можно получить оптическое свойство эллипса. Можно вывести уравнение касательной к эллипсу и обобщить для кривых второго порядка.

\subsection{Общее решение системы линейных алгебраических уравнений. Теорема Кронекера -- Капелли.}

Понятно, что нужно. Самое главное -- доказать теорему о ФСР и структуре решений СЛАУ. Ну и собственно доказать теорему Кронекера -- Капелли. 

Дополнительно ничего не требуется (можно сформулировать теорему Фредгольма, но док-во не нужно).

\subsection{Линейное пространство, базис и размерность. Линейное отображение конечномерных пространств, его матрица. Ядро и образ линейного отображения.}

Проблема билета в том, что очень много нужно определять. Структура ответа по билету:
\begin{enumerate}
    \item необходимо доказать корректность определения базиса (в любых двух базисах одинаковое количество векторов);
    \item ввести понятие линейного отображения. Столбцами матрицы линейного отображения являются образы координатных столбцов базисных векторов исходного пространства в координатах пространства образа;
    \item дать определение ядра, доказать теорему о сумме размерностей: размерность исходного пространства равна сумме размерностей ядра и образа.
\end{enumerate}

\subsection{Собственные значения и собственные векторы линейных преобразований. Диагонализируемость линейных преобразований.}

Дать определения линейных преобразований, их собственных значений и векторов. Отсюда \textit{обязательно получить} характеристическое уравнение и упомянуть, что характеристический многочлен не зависит от преобразования. 

Можно сослаться на формулу изменения матрицы при переходе к новому базису (также это можно сделать в предыдущем билете по необходимости).

Далее имеет смысл сказать, что, если есть базис из собственных векторов, то матрица диагонализуема. Размерность собственного подпространства не больше кратности корня характеристического уравнения.

Необходимое условие диагонализируемости матрицы преобразования -- вещественность всех собственных значений.

Достаточное условие диагонализируемости матрицы преобразования -- все собственные значение вещественные, и их кратность равна 1.

Критерий диагонализируемости матрицы преобразования -- все собственные значения вещественные и собственному значению кратности $k$ соответствует собственное подпространство размерности $k$.

Билет объемный, поэтому важно все сформулировать, про док-ва -- как успеете.

\subsection{Самосопряженные преобразования евклидовых пространств, свойства их собственных значений и собственных векторов.}

Нужно знать определение самосопряженного оператора, евклидово пространства. Желательно помнить, что преобразование является самосопряженным тогда и только тогда, когда его матрица в любом ОНБ симметрична.

Теоремы:
\begin{itemize}
    \item все собственные значения у самосопряженного преобразования вещественны;
    \item собственные подпространства соответствующие различным собственным значениям попарно ортогональны;
    \item если существует инвариантное подпространство самосопряженного преобразования, то его ортогональное дополнение тоже инвариантно;
    \item существует ортонормированный базис из собственных векторов самосопряженного преобразования.
\end{itemize}

Последняя теорема -- это мегатрон из предыдущих теорем.

Иногда последнее утверждение дается в виде критерия, где переход в обратную сторону почти очевиден: если есть ОНБ из собственных векторов, то там матрица диагональная, значит симметричная, значит преобразование самосопряженное.

В качестве доп. вопросов дают либо качественную задачу, либо какие-то формулировки (не обязательно по билету). (Либо могут спросить вообще что угодно.)

\subsection{Приведение квадратичных форм в линейном пространстве к каноническому виду. Положительно определенные квадратичные формы. Критерий Сильвестра.}

Приведение к каноническому виду осуществляется либо методом Лагранжа, либо операциями с матрицей квадратичной формы. Олег Константинович рекомендует именно второй способ: при первом необходимо не только обосновать, почему ваши преобразования -- это замена базиса (это сделать не так сложно), но обязательно еще и показать, почему мы в итоге получим каноническую форму.

Преобразования с матрицей делаются по формуле $B' = S^TBS$, что означает то, что все операции, проделанные со строками, проделываются также со столбцами. 

Этот способ лучше еще и потому, что он способствует ответу на два других вопроса в билете: критерий Сильвестра также доказывается с помощью элементарных операций. Для его доказательства потребуются два тонких факта.

Первый: на диагонали матрицы КФ стоят значения базисных векторов, поэтому в любом ОНБ они положительные, значит, не встретится особых случаев (при прибавлении нижних строчек в методе Гаусса к верхним может изменится значение минора). 

Предлагается не писать критерий Сильвестра для отрицательно определенной формы, чтобы спровоцировать комиссию задать о нём вопрос, на который вы с легкостью ответите.

\subsection{Дополнительно.} 

Нужно понимать, что могут спросить не только эти темы. Следует быть готовым рассказать про матрицу Грамма, понимать базовые вещи про поверхности второго порядка, помнить, что такое процесс ортогонализации Грамма -- Шмидта, и прочее.

\subsection{Разбор задач к письменной части.}

\begin{task}
    Дана прямая $l$
    \begin{equation*}
        \begin{pmatrix}
            x \\ y \\ z
        \end{pmatrix} = \begin{pmatrix}
            -1 \\ 0 \\ 1
        \end{pmatrix} + \begin{pmatrix}
            1 \\ 2 \\ -1
        \end{pmatrix}t
    \end{equation*}
    и плоскость $2x - y + z = 1$. Найти $\Pi: l \in \Pi, \Pi \perp \alpha$.
\end{task} 

Из уравнения прямой берем направляющую $\vecl{a} = (1,  2, -1)^T$, что дает нам из условия перпендикулярности находим нормаль к искомой поверхности $vecl{a}_2 = (2, -1, 1)^T$. По двум векторам легко пишется уравнение плоскости:
\begin{equation*}
    \vecl{r} = \vecl{r}_0 + \vecl{a}_1 l_1 + \vecl{a}_2 l_2.
\end{equation*}
На этом задача решена.

\begin{task}
    Найти каноническое уравнение прямой, которая проходит через точку $(1, 1, -2)$ и параллельна плоскостям $x + 3y - z = 1$ и $2x + y + 1 = -1$.
\end{task}

По задаче, когда-то решенной на лекции, известно, что даже не в ОНБ параллельный вектор можно найти по правилу векторного произведения: $\vecl{n} = [\vecl{n}_1, \vecl{n}_2] = (4, -3, -5)^T$. Если это знать, то ссылаться на задачу необязательно: достаточно найти этот вектор и проверить, что он параллелен обеим плоскостям.

Осталось записать уравнение:
\begin{equation*}
    \frac{x - 1}{4} = \frac{y - 1}{-3} = \frac{z + 2}{-5}.
\end{equation*}

\begin{task}
    Найти ортогональную проекцию точки $A(1, -4, -4)$ на прямую
    \begin{equation*}
        \frac{x - 2}{1} = \frac{y + 1}{1} = \frac{z - 1}{-3}.
    \end{equation*}
\end{task}

По уравнению прямой находим точку на прямой $M(2, -1, 1)$ и направляющий вектор $\vecl{a} = (1, 1, -3)^T$. Есть несколько вариантов решения, рассмотрим интересный: проведем через точку $A$ плоскость перпендикулярно прямой. Её уравнение $x + y - 3z - 9 = 0$. Остается найти точку пересечения прямой и плоскости. Прямую задаем в параметрическом виде:
\begin{equation*}
    \begin{cases}
        x = 2 + t, \\ y = -1 + t, \\ z = 1 - 3t,
    \end{cases}
\end{equation*}
подставляем в уравнение плоскости, получаем ответ.

\begin{task}
    Дана точка $C(3, -4, -2)$ в ОНБ. Требуется найти проекцию этой точки на плоскость $\pi$, которая задана как пересечение двух прямых:
    \begin{equation*}
        \frac{x - 5}{13} = \frac{y - 6}{1} = \frac{z + 3}{-4}
    \end{equation*}
    и
    \begin{equation*}
        \begin{cases}
            x - 13y + 3z = 0, \\
            4y + z - 9 = 0
        \end{cases}.
    \end{equation*}
\end{task}

Для первой прямой находим точку на прямой $A(5, 6, -3)$ и направляющий вектор $\vecl{a}(13, 1, -4)^T$. Чтобы задать плоскость, достаточно найти точку на второй прямой, не принадлежащую первой. Например, подходит точка $B(2, 3, -3)$. В качестве второго вектора берем $\vecl{a}_2 = \vecl{AB} / 3 = (1, 1, 0)^T$, отсюда получаем уравнение плоскости
\begin{equation*}
    x - y + 3z + 10 = 0.
\end{equation*}
Для нахождения проекции мы берем прямую, проходящую через данную точку перпендикулярно найденной плоскости. Получаем параметризацию
\begin{equation*}
    \begin{cases}
        x = 3 + t \\ y = -4 - t \\ z = -2 + 3t
    \end{cases}
\end{equation*}
откуда проекция $(2, -3, -5)$.

\textit{Примечание: этот метод \textbf{чудесен} еще и потому, что, если требуется найти симметричную точку, то делать необходимо все то же самое, только на последнем шаге удвоить $t$.}

\begin{task}
    Дана квадратичная форма $\kappa = -2x_2x_3 + 2x_1x_3 - x_2^2$. Записать ее в базисе $\vecl{e}_1' = -2\vecl{e}_3, \, \vecl{e}_2' = -\vecl{e}_1 + \vecl{e}_3, \, \vecl{e}_3' = 2\vecl{e}_1 - \vecl{e}_2$.
\end{task}

Приведем к каноническому виду двумя способами.

\textit{Способ 1: Лагранж одобряет.}
 \begin{equation*}
    \kappa = -2x_2x_3 + 2x_1x_3 - x_2^2 = -(x_2 + x_3)^2 + x_3^2 + 2x_1x_3 = -(x_2 + x_3)^2 + (x_1 + x_3)^2 - x_1^2 = -x_1'^2 + x_2'^2 - x_3'^2.
\end{equation*}

\textit{Способ 2: Олег Константинович доволен.}

\begin{equation*}
    \begin{pmatrix}
        0 & 0 & 1 \\ 0 & -1 & -1 \\ 1 & -1 & 0
    \end{pmatrix} \overset{\text{III -- II}}{\sim} \begin{pmatrix}
        0 & 0 & 1 \\ 0 & -1 & -1 \\ 1 & 0 & 1
    \end{pmatrix} \sim \begin{pmatrix}
        0 & 0 & 1 \\ 0 & -1 & 0 \\ 1 & 0 & 1
    \end{pmatrix} \overset{\text{I -- III}}{\sim} \begin{pmatrix}
        -1 & 0 & 0 \\ 0 & -1 & 0 \\ 1 & 0 & 1
    \end{pmatrix} \sim \begin{pmatrix}
        -1 & 0 & 0 \\ 0 & -1 & 0 \\ 0 & 0 & 1
    \end{pmatrix}.
\end{equation*}

На второй вопрос задачи тоже ответим двумя способами. 

Первый: чтобы выразить $x$ через $x'$, достаточно транспонировать матрицу, составленную из переходов от $e$ к $e'$:
\begin{equation*}
    \begin{cases}
        x_1 = -x_2' + 2x_3', \\ x_2 = -x_3', \\ x_3 = -2x_1' + x_2'.
    \end{cases}
\end{equation*}
Подставляем и получаем ответ.

Либо способ нумер два: записываем матрицу перехода
\begin{equation*}
    S = \begin{pmatrix}
        0 & -1 & 2 \\ 0 & 0 & -1 \\ -2 & 1 & 0
    \end{pmatrix}
\end{equation*}
откуда получаем матрицу квадратичной формы в новом базисе по формуле $B' = S^TBS$.

\begin{task}
    Дана система 
    \begin{equation*}
        \begin{cases}
            \alpha x_1 -3x_2 + 3x_3 = 0\\
            6x_1 + 2x_2 + x_3 = 0\\
            3x_1 + x_2 = 0
        \end{cases}.
    \end{equation*}
    Найти при каких условиях на $\alpha$ система имеет нетривиальное решение.
\end{task}

Выписываем матрицу:
\begin{equation*}
    \begin{pmatrix}
        \alpha & -3 & 3 \\ 6 & 2 & -1 \\ 3 & 1 & 0
    \end{pmatrix}
\end{equation*}
Система имеет ненулевое решение, если определитель равен нулю; отсюда $\alpha = -9$. Получаем матрицу
\begin{equation*}
    \begin{pmatrix}
        -9 & -3 & 3 \\ 6 & 2 & -1 \\ 3 & 1 & 0
    \end{pmatrix} \sim \begin{pmatrix}
        3 & 1 & -1 \\ 6 & 2 & -1 \\ 3 & 1 & 0
    \end{pmatrix} \sim \begin{pmatrix}
        0 & 0 & 1 \\ \hline 3 & 1 & 0
    \end{pmatrix}
\end{equation*}
откуда находим решение
\begin{equation*}
    \begin{pmatrix}
        x \\ y \\ z
    \end{pmatrix} = C\begin{pmatrix}
        1 \\ -3 \\ 0
    \end{pmatrix}.
\end{equation*}

\begin{task}
    Дано линейное преобразование $\varphi$ в базисе $\vecl{e}_1, \vecl{e}_2$ имеющее матрицу 
    \begin{equation*}
         A = \begin{pmatrix}
           4 & -2\\
           -1 & 5
        \end{pmatrix},
    \end{equation*}
    а
    \begin{equation*}
        \Gamma = \begin{pmatrix}
            2 & 1 \\ 1 & 3
        \end{pmatrix}
    \end{equation*} -- матрица Грамма.
    Найти собственные векторы. Являются ли собственные векторы ортогональными? Является ли преобразование самосопряженным?
\end{task}
Находим корни уравнения $\det(A - \lambda E) = 0$, получаем $\lambda_1 = 3$, $\lambda_2 = 6$. По ним находим собственные векторы $\vecl{h}_1 = (2, 1)^T$ и $\vecl{h}_2 = (1, -1)^T$.

Находим скалярное произведение:
\begin{equation*}
    (\vecl{h}_1, \vecl{h}_2) = h_1^T\Gamma h_2 = 0,
\end{equation*}
значит векторы ортогональные.

Можно ответить на третий вопрос двумя способами: либо сказать, что можно отнормировать найденные векторы и получить ОНБ из собственных векторов, значит, самосопряженное. Либо можно проверить равенство  (рукопожатия) $A^T\Gamma = \Gamma A$, которое также дает самосопряженность.

\begin{task}
    Дана квадратичная форма $x_1^2 + \alpha x_2^2 + \alpha x_3^2 - 2x_1x_2 - 2x_1x_3 + 4x_2x_3$. Записать матрицу квадратичной формы и найти все $\alpha$, при которых форма положительно определена.
\end{task}

Матрица квадратичной формы:
\begin{equation*}
   \begin{pmatrix}
         1 &-1 &-1\\
        -1 & \alpha & 2\\
        -1 & 2 & \alpha
   \end{pmatrix}.
\end{equation*}

Далее применяем критерий Сильвестра, который дает условия
\begin{equation*}
    \begin{cases}
        \Delta_1 = 1 > 0, \\ \Delta_2 = \alpha - 1 > 0, \\ \Delta_3 = \alpha^2 - 2\alpha > 0
    \end{cases},
\end{equation*}
что дает $\alpha > 2$.

\begin{task}
    Задана матрица преобразования в ОНБ
    \begin{equation*}
        \begin{pmatrix}
            8 & -3 & -3\\
            -3 & 8 & 3\\
            -3 & 3 & 8
        \end{pmatrix},
    \end{equation*}
    найти базис из собственных векторов.
\end{task}

Ищем сначала собственные значения. Из-за громоздкости вычислений сразу выпишем ответ: $\lambda_{1, 2} = 5, \lambda_3 = 14$. 

Для $\lambda_3$ соответствующий собственный вектор $\displaystyle \vecl{h}_3 = (-1, 1, 1)^T$, для $\lambda_{1, 2}$ -- $\vecl{h}_1 = (1, 1, 0)^T$ и $\vecl{h}_2 = (1, 0, 1)^T$.

Производим процесс ортогонализации Грамма-Шмидта:
\begin{equation*}
    \alpha = \frac{(\vecl{h}_1, \vecl{h}_2)}{(\vecl{h}_1, \vecl{h}_1)} = \frac{1}{2}.
\end{equation*}

Тогда $\vecl{h}_2' = \vecl{h}_2 - \alpha \vecl{h}_1$, затем нормируем векторы и получаем ответ.

\begin{task}
    Даны два пространства $M$ и $N$. Базис $M$ состоит из векторов $(1, t + t^2, t^2 + t^3, t^3)$. Базис в $N$ состоит из векторов $\set{\vecl{e}_1, \vecl{e}_2, \vecl{e}_3} = \set{2, t, 3t^2}$. Отображение $\varphi: M \to N$ -- дифференцирование. Исследовать отображение.
\end{task}

Найдем матрицу отображения. Для этого найдем образы базиса $e$ в $f$. 

\begin{equation*}
    \begin{cases}
        \varphi(\vecl{e}_1) = 0, \\ \varphi(\vecl{e}_2) = 1 + 2t, \\ \varphi(\vecl{e}_3) = 2t + 3t^2, \\ \varphi(\vecl{e}_4) = 3t^2.
    \end{cases}
\end{equation*}
что дает матрицу 
\begin{equation*}
    \Phi = \begin{pmatrix}
        0 & 1/2 & 0 & 0\\
        0 & 2 & 2 & 0\\
        0 & 0 & 1 & 1
    \end{pmatrix}.
\end{equation*}

Ядро является линейной оболочкой вектора $1$, образ -- все пространство, отображение сюръективным, но не инъективным.

\begin{task}
    Дана квадратичная форма $k(x) = 6x_1x_2 + 3x_3$ в ОНБ. Найти ОНБ, в котором эта форма имеет диагональный вид, написать этот вид, исследовать форму на знакоопределенность.
\end{task}

Матрица квадртичной формы
\begin{equation*}
    B = \begin{pmatrix}
        0 & 3 & 0 \\ 3 & 0 & 0 \\ 0 & 0 & 3
    \end{pmatrix}
\end{equation*}
Ищем собственные $\lambda_{1, 2} = 3$, $\lambda_3 = -3$. Находим собственные вектора: $\vecl{h}_3 = (1, -1, 0)^T$, $\vecl{h}_1 = (1, 1, 0)^T$, $\vecl{h}_2 = (0, 0, 1)^T$.

Форма в нужном ОНБ имеет вид
\begin{equation*}
    k(x') = 3x_1'^2 + 3x_2'^2 - 3x_3'^2
\end{equation*}
и знаконеопределена. Базис получился ортогональным, поэтому достаточно его отнормировать и получить желаемый ОНБ.

Матрица перехода к этому ОНБ имеет вид
\begin{equation*}
    S = \begin{pmatrix}
        \frac{1}{\sqrt 2} & 0 & \frac{1}{\sqrt 2} \\ \frac{1}{\sqrt 2} & 0 & -\frac{1}{\sqrt 2} \\ 0 & 1 & 0
    \end{pmatrix}.
\end{equation*}

\begin{task}
    Дан ОНБ и задана поверхность $(x - y - 3)(x + y + 1) = 5z$. Требуется найти прямолинейные образующие поверхности, проходящие через точку $A(-1, 1, -1)$.
\end{task}

Для решения задачи даже не требуется определять тип поверхности. Запишем параметрическое представление прямолинейной образующей:
\begin{equation*}
    \begin{cases}
        x = -1 + a_1 t \\
        y = 1 + a_2 t\\
        z = -1 + a_3 t
    \end{cases}.
\end{equation*}
Подставим это в исходное уравнение и получим
\begin{equation*}
    (-5 + (a_1 - a_2)t)(1 + (a_1 + a_2)t) = 5(- 1 + a_3t).
\end{equation*}
Раскрываем скобки и приравниваем коэффициенты при равных степенях. Получаем
\begin{equation*}
    \begin{cases}
        a_1^2 - a_2^2 = 0, \\ -4a_1 - 6a_2 = 5a_3,
    \end{cases}
\end{equation*}
откуда имеем две возможные системы:
\begin{equation*}
    \begin{cases}
        a_1 = a_2, \\ -10a_1 = 5a_3
    \end{cases} \implies \begin{cases}
        a_1 = a_2, \\ a_3 = -2a_1,
    \end{cases}
\end{equation*}
значит, в качестве направляющего вектора можно взять $(1, 1, -2)^T$;
и
\begin{equation*}
    \begin{cases}
        a_1 = -a_2, \\ 2a_1 = 5a_3
    \end{cases} \implies
    \begin{cases}
        a_1 = -a_2, \\ a_3 = 2a_1/5,
    \end{cases}
\end{equation*}
значит, в качестве направляющего вектора можно взять $(5, -5, 2)^T$.

Остается написать канонические уравнения прямых, что сделать очень легко.

\begin{task}
    Есть Евклидово пространство с ОНБ. Подпространство $L$ задано системой уравнений с матрицей
    \begin{equation*}
        \begin{pmatrix}
            1 & 1 & 1 & -17 \\ 2 & -13 & 0 & 1
        \end{pmatrix}
    \end{equation*}
    Найти базис в $L^{\perp}$.
\end{task}

Заметим, что каждая из строк матрицы системы, взятая в качестве вектора, является ортогональной к любому вектору из $L$ по определению. Таким образом, эти два вектора образуют базис в $L^{\perp}$. Размерности сходятся, задача решена.










