\section{Консультация Н.А.Гусева}

\subsection{Теорема Больцано -- Вейерштрасса и Критерий Коши сходимости числовой последовательности.}

После функана кажется очевидным.

\subsection{Ограниченность функции, непрерывной на отрезке.}

Стоит сказать что-то про связность. 

\subsection{Теорема о промежуточных значениях непрерывной функции.}
Настолько очевидный билет, что мы вообще забыли его затехать.

\subsection{Теоремы о среднем Ролля, Лагранжа и Коши для дифференцируемых функций.}

Нужно уметь доказывать все теоремы.

\paragraph*{Возможный вопрос:} а для этой функции работает та или иная теорема?

Пример: работает ли теорема Коши для вектор-функции $f: [a, b] \to \R^m$. Используются предположения о дифффиренцируемости функции на $(a, b)$.

В случае $m = 1$ теорема звучит так: существует $\xi \in (a, b)$ такое, что
\begin{equation*}
    f(b) - f(a) = f'(\xi).
\end{equation*}

В случае $m > 1$ это неверно! Можно вспомнить такую замечательную функцию из ТФКП: $f(t) = e^{it}$, контрпримером являются $a = 0$ и $b = 2\pi$.

Зато верно следующее утверждение:
\begin{equation*}
    \abs*{\frac{f(b) - f(a)}{b - a}} \leq \sup_{\xi \in (a, b)} \abs{f'(\xi)}.
\end{equation*}
Это \textit{теорема Лагранжа для векторной функции}. Тогда можно задать следующий вопрос: существует ли такое $\xi \in (a, b)$, что
\begin{equation*}
    \abs*{\frac{f(b) - f(a)}{b - a}} = \abs{f'(\xi)}?
\end{equation*}
Для непрерывных функци супремум достигается, конечно. В остальных случаях вопрос оставим без ответа.

\textit{Теорема Коробкова:} существуют числа $\xi_i \in (a, b)$ и $p_i \in [0, 1]$, $i \in \overline{1, m}$ такие, что $\sum p_i = 1$ и
\begin{equation*}
    \frac{f(b) - f(a)}{b - a} = \sum_{i = 1}^m p_if'(\xi_i).
\end{equation*}

Это более хорошее обобщение теоремы Лагранжа на многомерный случай. Эта теорема была доказана только в 2001 году!

\subsection{Формула Тейлора с остаточным членом в форме Пеано или Лагранжа.}

В случае разложения с остаточным членом в форме Пеано \textit{можно} использовать непрерывную дифференцируемость функции в условии (она необязательна, достаточно использовать просто $n$ раз дифференцируемость, но объем доказательства резко возрастает)

\subsection{Исследование функции одной переменной при помощи первой и второй производных.}

Здесь самое сложное доказательство -- доказательство выпуклости, лучше начать с него.

\subsection{Теорема о равномерной непрерывности функции, непрерывной на компакте.}

Стоит вспомнить лемму: $f$ -- равномерно непрерывна на промежутке $I$ тогда и только тогда, когда 
\begin{equation*}
    \forall \varepsilon > 0 \exists L > 0: \forall x, y \in I: \abs{f(x) - f(y)} \leq L\abs{x - y} + \varepsilon.
\end{equation*}

То есть равномерная непрерывность -- это \textit{почти что} липшицевость.

Пример НЕдостижимости знака -- Гёльдерово-непрерывные функции (напомним, что такие функции имеют хорошее приближение многочленом).

\subsection{Достаточные условия дифференцируемости функции нескольких переменных.}

Кажется, достаточно одного достаточного условия -- непрерывности частных производных.

\subsection{Теорема о неявной функции, заданной одним уравнением.}

Пусть дана функция $F(x, y)$ и точка $(x_0, y_0)$, для которой $F(x_0, y_0) = 0$ и $F_y(x_0, y_0)' = 0$, тогда существуют некоторые окрестности $U_{\delta}(x_0)$, $U_\varepsilon(y_0)$ и функция такая, что $f: U_{\delta}(x_0) \to U_{\varepsilon}(y_0)$, то есть $y = f(x)$, причем
\begin{equation*}
    f' = -\frac{F'_x}{F'_y}.
\end{equation*}

Обязательно нужно знать формулировки теорем об обратной функции и обратном отображении.

Если якобиан непрерывно дифференцируемого отображения $f: E \to G$, где открытые $E,G \in \R^n$, в точке обратим ($J(f(x)) \neq 0$), то отображение локально обратимо.

То есть существуют окрестности $U$ и $V$ такие, что $f$ отображает $U$ на $V$ взаимно однозначно, причем $f^{-1}$ непрерывно дифференцируемо и
\begin{equation*}
    Df^{-1}(y) = (Df(x))^{-1}.
\end{equation*}
В одномерном случае это равенство принимает вид
\begin{equation*}
    (f^{-1}(y))' = \frac{1}{f'(x)}.
\end{equation*}

\textit{Offtop.} Могут спросить теорему Гейне -- Бореля.

\subsection{Экстремумы функций нескольких переменных. Необходимые условия, достаточные условия.}

Необходимое условие \textit{обычного} экстремума: частные производные функции равны нулю.

Достаточное условие экстремума: матрица Гессиана положительно определена.

\paragraph*{Возможный вопрос:} а если матрица Гессиана отрицательно определена, будет экстремум?

В условном экстремуме мы максимизируем функцию $f$ при условиях $g_i = 0$ для $i \in \overline{1, m}$ (все функции лежат в классе $C^1$). Есть условие, про которое обычно забывают: 
\begin{equation}
    \rang Dg = m \tag{$**$} \label{fullrang}
\end{equation}
на $S = \set{x: g_i = 0 \, \forall i}$, где
\begin{equation*}
    g = \begin{pmatrix}
        g_1 \\ \vdots \\ g_m
    \end{pmatrix}.
\end{equation*}

Необходимые условия условного экстремума: если $x_0 \in S$ -- точка локального условного экстремума $f$ при выпуклых условиях связи, причем выполнено \eqref{fullrang}, то существуеют $\lambda_1, \ldots, \lambda_m$ такие, что $x_0$ -- стационарная точка функции Лагранжа
\begin{equation*}
    L = f - \sum_{i = 1}^n \lambda_ig_i.
\end{equation*}

Достаточные условия локального минимума: $d^2L > 0$ на касательном пространстве
\begin{equation*}
    \set{h \in \R^n\setminus \set{0} : Dg \cdot h = 0}.
\end{equation*}

\subsection{Свойства интеграла с переменным верхним пределом (непрерывность,
дифференцируемость). Формула Ньютона -- Лейбница.}

Могут спросить и про суммы Римана, и даже про критерий Дарбу (но последнее маловероятно).

\textit{Дифференцирование по верхнему пределу}: пусть $f \in C[a, b]$, тогда для любого $x \in [a, b]$ существует
\begin{equation*}
    \frac{d}{dx} \int_a^x f(t) dt = f(x)
\end{equation*}
(в концах отрезка подразумеваются односторонние производные).

Доказывается довольно просто через теорему о среднем через конструкцию
\begin{equation*}
    \frac{\int_a^{x + h} - \int_a^h}{h} = \frac{1}{h} \int_x^{x + h} f(x) dx.
\end{equation*}

\textit{Формула Ньютона -- Лейбница}: если $f \in C[a, b]$ и $F$ -- её первообразная, то
\begin{equation*}
    \int_a^b f(x) dx = F(b) - F(a).
\end{equation*}

Доказательство: пусть
\begin{equation*}
    \varphi(x) = \int_a^x f(t) dt.
\end{equation*}
По теореме о дифференцируемости по верхнему пределу это первообразная функции $f(x)$. Если $F$ -- другая первообразная функции $f$, то
\begin{equation*}
    (\varphi - F)' = 0 \implies \varphi - F = C.
\end{equation*}

\paragraph*{Вопрос:} а если существует первообразная, но $f \notin C[a, b]$, то будет ли верна формула Ньютона -- Лейбница? Это не такой простой вопрос, поэтому вряд ли его спросят.

Более реальный вопрос: $f \in C(\R)$, $\exists f'(0) > 0$. Будет ли $f$ монотонна в окрестности нуля?

\subsection{Равномерная сходимость функциональных последовательностей и рядов.
Непрерывность, интегрируемость и дифференцируемость суммы функционального
ряда.}

Самое важное -- когда ряд можно дифференцировать? Если ряд равномерно сходится в окрестности точки дифференцирования. 

Пусть $f_n \in C^1[a, b]$, такое что
\begin{enumerate}
    \item $\exists x_0 \in [a, b)$ такой, что $f_n(x_0)$ сходится;
    \item $f_n'$ сходится равномерно на $[a, b]$ к функции $\varphi$.
\end{enumerate}
Тогда $f_n' \rightrightarrows g$ на $[a, b]$, где $g \in C^1[a, b]$, и
\begin{equation*}
    g'(x) = \lim_{n \to \infty} f'_n(x) \forall x \in [a, b].
\end{equation*}
	
Пусть функциональная последовательность $f_n$ сходится к $f$. Имеем
\begin{equation*}
    \int_{x_0}^x f_n' (t) dt \rightrightarrows \int_{x_0}^x \varphi(t) dt
\end{equation*}
и 
\begin{equation*}
    f_n(x) - f_n(x_0) \to A.
\end{equation*}
Тогда
\begin{equation*}
    f_n(x) \to A + \int_{x_0}^x \varphi(t) dt \overset{def}{=} g,
\end{equation*}
$g' = \varphi \implies f'_n = g'$.

\paragraph*{Вопрос:} как доказать без непрерывной дифференцируемости?

Также могут спросить про признаки Дирихле и Абеля -- обязательно знать формулировки.

\subsection{Степенные ряды. Радиус сходимости. Бесконечная дифференцируемость суммы степенного ряда. Ряд Тейлора.}

После ТФКП дифференцируемость степенного ряда вопросов не вызывает (как и все остальные вопросы в билете в целом).

\paragraph*{Вопрос:} привести пример функции, которая имеет радиус сходимости бесконечность, но не раскладывается в ряд Тейлора. 

Тут можно рассмотреть $\exp(-1/x^2)$, которая имеет особенность в точке 0, поэтому и не раскладывается. 

\subsection{Формула Грина. Потенциальные векторные поля на плоскости.}

Формулировку для формулы Грина давать в общем случае, а доказательство достаточно в прямоугольнике.

Криволинейные интегралы вводятся через параметризацию, поэтому нужно сказать про независимость интеграла от выбора параметризации. 

Потенциальность равносильна независимости работы (криволинейного интеграла) от пути.

\subsection{Формула Остроградского -- Гаусса. Соленоидальные векторные поля.}

\textit{Формула Остроградского -- Гаусса}:
\begin{equation*}
    \iiint_G \Div (\vec{w}) dxdydz = \oiint_{\partial G} \vec{w}\vec{n} ds.
\end{equation*}

Пусть есть поверхность, заданная уравнением $z(x, y)$. В качестве области $G$ выступает подграфик этой функции. Достаточно доказать для $G = \set{z < f(x, y)}$ и носитель $w$ -- компакт.

Доказывали мы это через функцию среза $\chi (x)$:
%ебейшая картинка бля

Рассмотрим также фунцкию
\begin{equation*}
    \chi_h(x, y, z) = \chi\delim*{\frac{f(x, y) - z}{h}}
\end{equation*} 

\textit{К сожалению, это особо затехать не получилось, потому что тут картинки.}

\textit{Лемма.} Если носитель векторного поля компакт, то 
\begin{equation*}
    \iiint_{\R^3} \Div \vec{w} dxdydz = 0.
\end{equation*}

\textit{Доказательство:} запишем
\begin{equation*}
    \Div \vec{w} = \frac{\partial w_1}{\partial x} + \frac{\partial w_2}{\partial y} + \frac{\partial w_3}{\partial z},
\end{equation*}
а
\begin{equation*}
    \int_{-\infty}^{+\infty} \frac{\partial w_1}{\partial x} dx = 0
\end{equation*}
по формуле Ньютона -- Лейбница.

Тогда
\begin{equation*}
    0 = \iiint_{\R^3} \Div(\vec{w}, \chi_h) dxdydz = \iiint_{\R^3} \vec{w} \nabla \chi_h dxdydz + \iiint_{\R^3} \chi_h \Div \vec{w} dxdydz.
\end{equation*}
Последний интеграл
\begin{equation*}
    \iiint_{\R^3} \chi_h \Div \vec{w} dxdydz \to \iiint_{G} \Div \vec{w} dxdydz.
\end{equation*}
Далее,
\begin{equation*}
    \nabla \chi_h = \frac{1}{h}\chi'\delim*{\frac{f(x, y) - z}{h}}\cdot\begin{bmatrix}
        f'_x \\ f'_y \\ -1
    \end{bmatrix}.
\end{equation*}

Наконец,
\begin{equation*}
    \int_{f(x, y) - h}^{f(x, y)} \frac{1}{h}\chi'\delim*{\frac{f(x, y) - z}{h}}dz = \int_{f(x, y) - h}^{f(x, y)} -\frac{\partial}{\partial z}\chi\delim*{\frac{f(x, y) - z}{h}}dz = 1.
\end{equation*}

\paragraph*{Вопрос:} пусть $\rot \vec{w} = 0$ и $\Div \vec{w} = 0$. Что можно сказать про векторное поле $\vec{w}$? Ответ: $\vec{w} = \nabla \varphi$ и $\Delta \varphi = 0$.

\subsection{Формула Стокса.}
<<Это то же самое, что формула Остроградского -- Гаусса.>> \textcopyright

\textit{Формула Стокса:}
\begin{equation*}
    \iint_S \rot \vec{A}d\vec{S} = \int_{\Gamma} \vec{A} d\vec{l},
\end{equation*}
причем ориентация границы должна быть согласованна с ориентацией поверхности.

Доказательство достаточно для $\vec{A} = (A, 0, 0)^T$ через параметризацию и формулу Грина.

\subsection{Достаточные условия сходимости тригонометрического ряда Фурье в точке.}

\textit{Признак Дини}: пусть есть абсолютно интегрируемая $f \in L^1(-\pi, \pi)$ и дана точка $x \in (-\pi, \pi)$ и выполнено условие Дини: сущесвует $a \in \R$ такое, что
\begin{equation*}
    \int_0^{\delta} \frac{\abs{f(x + z) + f(x - z) - 2a}}{z} dz < \infty.
\end{equation*}
Тогда 
\begin{equation*}
    S_n[f](x) \to a,
\end{equation*}
где
\begin{equation*}
    S_n[f](x) = \sum_{k = -n}^n c_k e^{ikx}
\end{equation*}
и
\begin{equation*}
    c_k = \frac{1}{2\pi}\int_{-\pi}^{\pi} f(x) e^{ikx} dx.
\end{equation*}

Частичные суммы ряда Фурье можно представить в виде свертки с ядром Дирихле:
\begin{equation*}
    S_n[f] (x) = \int f(x + z) D_n(z) dz,
\end{equation*}
где $D_n$ -- ядро Дирихле:
\begin{equation*}
    D_n(x) = \frac{1}{2}\sum_{k = -n}^{n} e^{ikx}.
\end{equation*}

Тогда
\begin{equation*}
    S_n[f](x) - f(x) = \int(f(x + z) - f(x)) D_n(z) dz.
\end{equation*}
Отсюда следует
\begin{equation*}
    \int_0^{+\infty} \frac{f(x + z) + f(x - z) - 2a}{z}D_n(z) dz.
\end{equation*}
Остается вспомнить представление ядра
\begin{equation*}
    D_n(z) = \frac{\sin\delim*{n + \frac12}z}{2\sin \frac z2}.
\end{equation*}
С помощью математики и магии получаем
\begin{equation*}
    \int \varphi(z) \sin nz dz \to 0,
\end{equation*}
откуда по лемме Римана -- Лебезгуя $\varphi \in L^1$.

Нужно знать формулировку теоремы об осциляции Лебега (док-во вряд ли попросят).

\subsection{Достаточные условия равномерной сходимости тригонометрического ряда Фурье.}

Функция должна быть $2\pi$-периодическая и дважды непрерывно дифференцируема. То есть ряд $\sum c_k$ должен сходиться абсолютно, а это выполнено при $c_k = \mathcal{O}(1/k^2)$, откуда сразу следует дважды непрерывная дифференцируемость из интегрирования по частям. 

\subsection{Непрерывность преобразования Фурье абсолютно интегрируемой функции.
Преобразование Фурье производной и производная преобразования Фурье.}

Преобразование Фурье 
\begin{equation*}
    F[f](\xi) = \frac{1}{\sqrt{2\pi}} \int f(x) e^{-i\xi x} dx
\end{equation*}
является непрерывным по $\xi$ как интеграл, зависящий от параметра по теореме о непрерывности интеграла, зависящего от параметра. Или можно воспользоваться теоремой Лебезгуя об ограниченной сходимости. Если вы хотите, можно и другие теоремы использовать. 

Но Гусев очень просит теорему об ограниченной сходимости.

Пожалуйста.

Преобразование Фурье производной легко вспомнить, проинтегрировав по частям:
\begin{equation*}
    F[f'](\xi) = i\xi F[f](\xi).
\end{equation*}
Аналогично легко найти производную преобразования Фурье:
\begin{equation*}
    F[f](\xi)' = F_x[(-ix)f(x)](\xi)
\end{equation*}

\subsection{Дополнительные вопросы.}

Идея доказательства теоремы Вейерштрасса. 

Пусть есть $f \in C[0, 1]$. Рассмотрим многочлен Бернштейна:
\begin{equation*}
    P_n(t) = \sum_{k = 0}^n f\delim*{\frac{k}{n}}C_n^k t^k(1 - t)^{n - k}.
\end{equation*}

Утверждение. $P_n \rightrightarrows f$ на $[0, 1]$ при $n \to \infty$.

Зафиксируем $t \in [0, 1]$. Рассмотрим $\displaystyle f\delim*{\frac 1n S_n}$. Пусть $\PP(\set{x_n = 1}) = t$, $\PP(\set{x_n = 0}) = 1 - t$ и
\begin{equation*}
    S_n = \sum_{k = 1}^n x_k.
\end{equation*}
Тогда
\begin{equation*}
    \EE\delim*{f\delim*{\frac 1n S_n}} = P_n(t).
\end{equation*}

Имеем $f \in C[0, 1]$, значит $f$ равномерно непрерывна на $[0, 1]$:
\begin{equation*}
    \forall \varepsilon > 0 \exists \delta > 0: \abs{t - t'} < \delta \implies \abs{f(t) - f(t')} < \varepsilon.
\end{equation*}

Разобьем многочлен Бернштейна на две суммы:
\begin{equation*}
    P_n(t) = \sum_{k: \, \abs{\frac kn - t} \leq \delta} + \sum_{k: \, \abs{\frac kn - t} > \delta}.
\end{equation*}

Имеем
\begin{equation*}
    \abs{f(t) - P_n(t)} \leq \sum_{k = 0}^n \abs*{f(t) - f\delim*{\frac{k}{n}}}C_n^k t^k(1 - t)^{n - k} = \sum_{k: \,  \abs{\frac kn - t} \leq \delta} + \sum_{k: \, \abs{\frac kn - t} > \delta}. 
\end{equation*}

Первая сумма
\begin{equation*}
    \sum_{k: \abs{\frac kn - t} \leq \delta} \abs*{f(t) - f\delim*{\frac{k}{n}}}C_n^k t^k(1 - t)^{n - k} \leq \varepsilon \sum_{k = 0}^n C_n^k t^k(1 - t)^{n - k} = \varepsilon,
\end{equation*}
то есть маленькая.

Будем считать, что $f$ ограниченна константой $c$. Тогда вторая сумма
\begin{equation*}
    \sum_{k: \, \abs{\frac kn - t} > \delta} \abs*{f(t) - f\delim*{\frac{k}{n}}}C_n^k t^k(1 - t)^{n - k} \leq 2c\cdot \sum_{k: \, \abs{\frac kn - t} > \delta} C_n^k t^k(1 - t)^{n - k}.
\end{equation*}
С точки зрения вероятностей
\begin{equation*}
    C_n^k t^k(1 - t)^{n - k} = \PP\delim*{\set*{\abs*{\frac{S_n}{n} - t} > \delta}} \leq \frac{\mathbb{D} \frac{S_n}{n}}{\delta^2} = \mathcal{O}\delim*{\frac 1n}.
\end{equation*}
Последний переход в качестве упражнения.

\subsubsection{Интересные задачи.}
\begin{itemize}
    \item[(0)] Обосновать сходимость последовательности $x_n = \sin( \ldots (\sin 1))$.
    
    Имеет место красивое геометрическое доказательство (которое мы не будет техать). 
    \item[(1)] Дана матрица $A = A^T$. Исследовать $x^T A x \to \min, \max$ на единичной сфере.

    Существует два способа решения:
    \begin{itemize}
        \item \textit{Решение через линал.} Переходим к ОНБ из собственных векторов.
        \item \textit{Решение через матан.} Условие экстремума:
        \begin{equation*}
            L = x^TAx - \lambda (x^Tx - 1);
        \end{equation*}
        отсюда дифференцированием получаем $2Ax - 2\lambda x = 0$.
    \end{itemize}

    \item[(2)] Найти значение $\exp (A)$ для матрицы $\displaystyle A = \begin{pmatrix}
        1 & 1\\
        0 & 1
    \end{pmatrix}$.

    Стоит вспомнить, что $exp(A + B) = \exp(A) \exp(B)$ для коммутирующих матриц. Здесь же можно рассмотреть разложение на $\displaystyle \begin{pmatrix}
        1 & 0\\
        0 & 1
    \end{pmatrix} + \begin{pmatrix}
        0 & 1\\
        0 & 0
    \end{pmatrix}$.
    
    \item[(3)] Доказать $\det e^A = e^{\tr A}$.

    Доказательство: $d(\det A) = \tr (A^\# dA)$ -- формула Якоби, где $A^\#A = AA^\# = \det A \cdot E$.

    \item[(4)] У функции только одна стационарная точка, и это локальный минимум. Является ли эта точка глобальным минимумом?

    \item[(5)] Доказать $F[e^{-x^2/2}](\xi) = e^{-\xi^2 / 2}$. 
    
    Идея решения через ТФКП: считаем действительную часть интеграла: 
    \begin{equation*}
        \frac{1}{\sqrt{2\pi}}\int e^{tx}e^{-x^2 / 2}dx = e^{t^2 / 2}.
    \end{equation*}
    Если допустить, что $t = i\tau$ -- комплексное число, то функция в левой части -- аналитическая, и совпадает с правой частью на всей действительной прямой. Тогда по теореме о единственности она совпадает с этой функцией везде.
\end{itemize}



































































